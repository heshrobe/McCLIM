\chapter{Using command tables}
\label{using-command-tables}

A \emph{command table} is an object that is used to determine what
commands are available in a particular context and the ways in which
commands can be executed.

Simple applications do not manage command tables explicitly.  A
default command table is created as a result of a call to the macro
\texttt{define-application-frame} and that command table has the same
name as the application frame.

Each command table has a \emph{name} and that CLIM manages a global
\emph{namespace} for command tables.

\Defun {find-command-table} {name \key (errorp t)}

This function returns the command table with the name \textit{name}.
If there is no command table with that name, then what happens depends
on the value of \textit{errorp}.  If \textit{errorp} is \emph{true},
then an error of type \texttt{command-table-not-found} is signaled. 
If \textit{errorp} is \emph{false}, otherwise \texttt{nil} is
returned.

\Defmacro {define-command-table} {name \key inherit-from menu
  inherit-menu}

\Defun {make-command-table} {\\name \key inherit-from menu
  inherit-menu (errorp t)}

By default command tables inherit from \texttt{global-command-table}. A
command table inherits from no command table if \texttt{nil} is passed as an
explicit argument to \textit{inherit-from}.
